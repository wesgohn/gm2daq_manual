\subsection{Running Event Builder}

It is also useful to run the event builder, which assembles data fragments from all specified frontends and combines them into a single MIDAS event. This program is located in gm2daq/eventbuilderNew. After compiling, start the event builder using

\begin{verbatim}
./mevb -e GM2 -b BUF
\end{verbatim}

where again GM2 is the experiment name and BUF is the prefix used to identify each unique event buffer. The event buffers must be defined for each frontend in it's ODB Common block, and they must each start with BUF. For instance, we generally define the buffers as BUF01, BUF02, BUF03, etc. 

For each enabled frontend, the user may specify whether or not to send it's data to the event builder by setting an ODB flag under the Global settings for that frontend called "Send to Event Builder", i.e.

\begin{verbatim}
"/Equipment/AMC1301/Settings/Globals/Send to Event Builder" y
\end{verbatim}

\subsection{ODB Options}

For normal running, you should not need to change the ODB options. There are settings that can be used to enable or disable the building of data from individual frontends, but they have been made redundant by the 'send to event builder' flag in the globals settings for each frontend.