

\section{How To Run ROME}

\subsection{Download ROME}
\begin{itemize}
\item git clone ssh://p-gm2midas@cdcvs.fnal.gov/cvs/projects/gm2midas
\item cd gm2midas/rome
\item  git checkout master
\end{itemize}

\subsection{Define Environment Variables}

For bash
\begin{itemize}
\item export ROMESYS=/directory/where/you/have/rome
\item export LD\_LIBRARY\_PATH=\$ROOTSYS/lib:\$LD\_LIBRARY\_PATH
\end{itemize}

For csh, tcsh
\begin{itemize}
\item setenv ROMESYS /directory/where/you/have/rome
\item setenv LD\_LIBRARY\_PATH=\$ROOTSYS/lib:\$LD\_LIBRARY\_PATH
\end{itemize}

\subsection{Make the ROMEBuilder}
\begin{itemize}
\item cd \$ROMESYS
\item make
\end{itemize}

\subsection{Check out \& build Calorimeter ROME DQM}
\begin{itemize}
\item The first step is to checkout the code from the repository. 

git clone ssh://p-gm2dqm@cdcvs.fnal.gov/cvs/projects/gm2dqm-gm2-dqm

\item cd gm2dqm-gm2-dqm/calo/rome/histoGUI

\item To build: \$ROMESYS/bin/romebuilder.exe -i histoGUI.xml -midas

Use the above command after the first git pull, also after making any change in the histoGUI.xml.
Otherwise use \bf{make}.

\end{itemize}

\subsection{Run the DQM}
\subsubsection{Offline}
\begin{itemize}
\item ./hguiexample.exe -i romeConfig.xml -pi ./ -r 5450 -m offline
\item Replace 5450 by your run number.
\item To run in batch mode:
\item ./hguiexample.exe -i romeConfig.xml -pi ./ -r 5450 -m offline -b
\item To run with a specific event number:
\item ./hguiexample.exe -i romeConfig.xml -pi ./ -r 5450 -m offline -e EVENTNO

\end{itemize}

\subsubsection{Online}
\begin{itemize}
\item ./hguiexample.exe -i romeConfig.xml
\item The analyzer watches the status of the frontend. When you start a run (e.g. from odbedit), the analyzer will start analyzing the online data and ROOT output files will be created.

\item For online running, the following steerable params in /gm2dqm-gm2-dqm/calo/rome/histoGUI/romeConfig.xml must match your MIDAS experiment:

    \item Host - $<mserver host>:<mserver port>$
    \item Experiment - MIDAS experiment name (as defined in exptab)

\end{itemize}

\subsubsection{Running ROME remotely}

On the server side: in romeConfig.xml:
The program mode. Must be one of the following : 0 : analyzer; 1 : monitor; 2 : analyzer and monitor;\\
 3 : monitor connected to an analyzer.

$<ProgramConfiguration>\\
 <ProgramMode>0</ProgramMode>\\
 </ProgramConfiguration>\\
 <SocketServer>\\
 <Active>true</Active>\\
 <PortNumber>9091</PortNumber>\\
 </SocketServer>$
 
 Run command: ./hguiexample.exe -i romeConfig.xml
 
 On the client side:

$<ProgramConfiguration>$\\
 $ <ProgramMode>3</ProgramMode>\\
 </ProgramConfiguration>\\
  <SocketClient>$\\
  $ <!-- Host which runs the ROME analyzer. -->\\
  <Host>g2be1</Host>\\
  <!-- Port number for the socket connection. -->\\
  <Port>9091</Port>\\
  </SocketClient>$\\
  
  Run command: ./hguiexample.exe 
  
  \subsection{Miscellaneous}
  \begin{itemize}
  \item We can specify a range of event numbers from romeConfig.xml.\\
    $ <EventNumbers></EventNumbers>$
  \item We can skip events from romeConfig.xml.\\
    $<EventStep>1</EventStep$
  \item We can turn on/off tasks/tabs from romeConfig.xml, changing true to false and vice versa.\\
    
    $  <Task>\\
    <TaskName>FillHisto</TaskName>\\
    <Active>true</Active>\\
    </Task>\\
    $
  \item Update frequency can be changed from romeConfig.xml.\\
    $<UpdateFrequency>5000</UpdateFrequency>$
    
  \item To change the range of a histogram/graph you need to access histoGUI.xml.\\
    
    $<Tasks>\\
    <Task>\\
    <TaskName>FillHisto</TaskName>\\
    <Histogram>\\
    <HistName>h2_Exy</HistName>\\
    <HistAccumulate>false</HistAccumulate>\\
    <HistFolderName>CT</HistFolderName>\\
    <HistArraySize>1</HistArraySize>\\
    <HistType>TH2F</HistType>\\
    <HistXLabel>xseg</HistXLabel>\\
    <HistYLabel>yseg</HistYLabel>\\
    <HistXNbins>9</HistXNbins>\\
    <HistXmin>0.5</HistXmin>\\
    <HistXmax>9.5</HistXmax>\\
    <HistYNbins>6</HistYNbins>\\
    <HistYmin>0.5</HistYmin>\\
    <HistYmax>6.5</HistYmax>\\
    </Task>\\
    </Tasks>\\$
 \end{itemize}


